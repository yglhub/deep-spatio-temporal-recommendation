\section{Introduction}\label{sec:intro}

The prevalence of GPS-enabled device such as  smart phone leads to the thrift of many Location-Based Services (LBS) including GoogleMap and Foursquare, where users can access the information of millions of Point-of-Interests (PoIs) like restaurants and shopping centrers. More and more customers relies on such LBS to search for PoIs of their interest on a daily basis. An important type of PoI information users are looking for on such LBS is \textbf{PoI rating}, i.e., a ``score" that measures how good the PoI is in terms of service provided. For example, on GoogleMap, each PoI is rated on a scale of 1 to 5 ``stars". The more stars a PoI has, the higher it is rated, reflecting that the PoI offers services of high quality.

PoI rating plays a crucial role in customers' decision making process. Obviously, a user is more likely to use the service of PoIs with higher ratings, comparing with similar (in terms of location and price), but lower-rated PoIs. PoI rating is also an important factor that affects the rank of a PoI in the search result return to a user. Additonally, it is used to determine whether a PoI is to be recommended to users by many recommendation systems. For the above reasons, PoI rating is crucial to the owner a the PoI, and this rating also provides important insights on how to improve his business in order to attract more customers.

A fundamental requirement for PoI rating is it should accurately reflect the quality of service of the PoI. On LBS, PoI rating is usually automatically generated via a data mining process, which takes into consideration of several factors. Such factors usually include individual user's explicit rating and review of the PoI, location and category of the PoI, etc. (e.g., ). This rating generation method is similar to the product rating system for online shopping websites such as eBay and Amazon, which is known to be suffering from several problems, such as:

1. A major concern of this system is it is vulnerable to \textbf{shilling} attacks. Shilling attack refers to the behaviour of giving fake ratings and reviews to products, in order to manipulate their ratings. Its effectiveness has been shown by several studies on user-rating based product rating systems. Due to the aforementioned reasons, PoI owners strive to improve their ratings and some of them may resort to this attack. It is estimated that -\% of user rating on ... are likely to be fake. Since anyone can rate PoIs on LBS such as GoogleMaps, there is no reason to believe that their PoI rating systems are immune from such attacks. 

2. This system relies on complex text-mining algorithms. Written review is usually more trustworthy and provides more information about a product comparing with simple a rating score. However, extract useful rating information (e.g., positive/negative labels) from written reviews is known to be a challenging task. Although a few methods has been proposed for automatic review-text mining, the mining process is still expensive and the accuracy is not very satisfactory. More importantly, when used for PoI rating generating, the errors from multiple user's review will accumulate. 

3. The system requires extra user efforts and may not be objective. It require users to explicitly give their ratings to a location and/or write reviews, which requires extra user efforts and may not be objective (e.g., a ``score" can be interpreted very differently by different users, inaccurate description, only users who strongly dislike or like the PoI are motivated to rate or write reviews, etc.). 

The root of the above problem is that this method strongly relies on individual user's explicit ratings and written reviews, but their actual behaviours. However, a unique difference between a PoI and a normal product is that for a PoI, an LBS can not only know who visited it, but also knows the trajectories of these visitors. This is made possible by either keep tracking user's locations, or ask users to voluntarily report check-in to the PoI. For example, the publicly available Foursquare dataset contains more than 3 millions of user self-reported check-ins.

These trajectories provides a unique dimension that could be leverage for PoI rating, which provides a possible solution to the aforementioned problems of normal product rating method. To this end, we study the spatio-temporal PoI rating prediction problem. We aim to answer the following question: \textit{Is it possible to accurately predict the rating of a PoI by mining the trajectories of its historical visitors?} We want to point out that our problem is different from user-location rating prediction, where the goal is to predict whether \textit{a specific user} likes a PoI, but not to predict the overall rating of a PoI. The former finds its applications in making personalized recommendations, while the later is crucial for the aforementioned applications.

There are several potential advantages of using trajectory data for PoI rating: (i) It is prone to alteration by fake-user-ratings and spam/bot-user-ratings. It is much harder and more expensive to forge a valid trajectory than to submit a fake rating score or review. As such, it is resistant to common shilling attack. {ii} It does not require an intermediate process to extract label information from written reviews. (iii) no extra user effort to capture their opinions such as filling in rating forms or submit a written review, and (iv) it is based on user's behaviour, thus could be more objective.

A straightforward way to predict the rating of a PoI is to simply count the number of visitors. This is based on the assumption that high-rating PoIs are usually popular locations since they are more attractive to costumers. Although the assumption is reasonable to some extend, it has some obvious drawbacks. First, a PoI being popular could only be the result of its location. For example, a Starbucks coffee shop located in a busy airport could have much more visitors than another Starbucks in suburb area, despite that the products they serve is largely the same. Second, user's choice of PoIs may be affected by time. For example, a customer who is on the way to work is likely to choose a restaurant for breakfast only because it is close to work and serves food fast, not because the food is exceptional. Third, the visits from different user may different predictive power since some user may be more selective in choosing PoIs in certain circumstance. For example, a user who is more familiar with an area may knows better which restaurant have the best service.  


To address the above challenges, we propose a \textit{visiting-pattern-based PoI rating prediction framework} that captures the spatio-temporal characteristic of visits. The visiting-pattern is extracted from a user's trajectory, which reflects how often the user visits a PoI, the visiting time, and the travel distance. The proposed framework is based on a comprehensive analysis of the intuitive relation between user visits to a PoI and its rating over the Foursquare dataset. The core the proposed framework is a novel regression model that takes into consideration the difference between visitors, visiting time, PoI locations, etc. We show in experiments on real world dataset that the proposed method outperforms not only the simple counting based method, but also the method based on state-of-the-art user-location rating techniques. We summarize our contributions as follows:
\begin{itemize}
\item We formally define the spatio-temporal PoI rating problem, and conduct the first systematic study of the problem.

\item We perform a comprehensive analysis of the Foursquare check-in dataset, in order to identify relations between user visits and PoI ratings.

\item Based on our analysis, we propose a visiting-pattern-based PoI rating prediction framework. It captures the spatio-temporal characteristic of visits by extracting patterns from user's trajectories. The core the proposed framework is a novel regression model that takes into consideration the difference between visitors, visiting time, PoI locations, etc. 

\item The proposed technique is implemented and evaluated on real world dataset. The results show it outperforms not only the simple counting based method, but also the method based on state-of-the-art user-location rating techniques.
\end{itemize}


The rest of this work is organized as follows. We survey the state-of-the-art on location rating in Section \ref{}. Then, we formally define the problem spatio-temporal PoI rating in Section \ref{}. Section present our analysis of the Foursquare check-in dataset. Details of the proposed framework is described in Section \ref{}. Our solution is evaluated in Section \ref{}. And finally, we conclude our work in Section \ref{}.  