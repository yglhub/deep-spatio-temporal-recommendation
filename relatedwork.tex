\section{Related Work}\label{sec:rw}

The problem studied in this paper is related to user-location rating prediction, which is a major research issue in location recommendation systems. Unlike the PoI rating discussed in this paper, user-location rating is a rating a specific user gives to a location to reflect his personal feeling towards the location. 

In general, there are two types techniques of user-location rating. 1) Classification-based techniques. It takes a user-location pair as input and label it with one of two classes: ``like" or ``dislike". Non-binary classifier can also be built model the case where users have more than two rating options. But most commonly, the interest is only to determine if the user will like the location or not, not to predict the ``degree" of user's preference of the PoI. 2) Ranking-based techniques. The input is a user and a list of candidate locations, and the goal is to rank the candidate locations by the probability that the user likes the location. Ranking-based technique is also referred to as check-in prediction in some literature, to emphasis that the goal is to predict which location(s) the user is most likely to visit next given his current trajectory.

General user-product rating prediction techniques can be directly used for user-location rating by simply treating locations as products. One of the most widely used technique is Matrix Factorization, which makes predictions by mapping both users and locations into a set of latent features. The features are leant from known user-location ratings in the training dataset. However, generic techniques such as Matrix Factorization do not take into consideration spatio-temporal characteristic of locations or user movements. A series of spatio-temporal user-location rating prediction techniques have then been proposed to improve the prediction performance by leverage spatio-temporal data. 


It is worth mentioning that user-location rating prediction can also be used to address our problem if combined with traditional user-rating based method: Since the overall PoI rating is usually derived from individual ratings and reviews of the PoI, we can first predict the individual ratings of the PoI based on the spatio-temporal information of its visitors, and then generate the PoI rating based on the predicted individual user-location ratings (e.g., take the average visitor rating as overall PoI rating like some LBS). However, the prediction error introduced in the user-location rating prediction process will accumulate in the final prediction results. This is supported by our experiment, which shows even the state-of-the-art user-location rating prediction technique cannot provide good accuracy for PoI rating comparing with our technique as a result of this error accumulation. 