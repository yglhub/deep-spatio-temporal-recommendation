\section{Spatio-Temporal Trend Dissemination Rule Mining}\label{sec:mining}
This section describes our approach at mining spatio-temporal trend
dissemination rules. As a first step, we need to acquire past trends, to mind
dissemination rules from. For this purpose, we apply existing textual trend
mining solutions proposed in the recent past, which are briefly sketched in
Section \ref{subsec:Signi} for self-containment. Next, as a second step used for
preprocessing, we employ a space composition scheme in Section
\ref{subsec:decomp} to ensure having a similar number of tweets in each spatial
region using a k-d tree.
As a third step, we model the flow of trends over space and time in Section
\ref{subsec:tensor}. Therefore, we transform a \emph{trend count matrix}, as
defined in Definition \ref{def:trendcountmatrix}, into a \emph{trend flow
tensor}, which describes the flow from any source region to any target region at any
point in time for any trend. Consequently, constructing a \emph{trend flow
tensor} for each trend that we mined in the first step, yields a four-mode Space
$\times$ Space $\times$ Time $\times$ Trends tensor, which will be fed to our
fourth step, the mining step. In the mining step proposed in Section
\ref{subsec:mining}, we employ a tensor factorization approach to discover
latent features of trends, latent features of trend-source-regions and latent
features of trend-target-regions. These latent features allow us to cluster
trends into sets of trends which disseminate similarly over space and time.
Then, for each cluster of similar trends, we obtain trend flows from the
\emph{reconstructed trend flow tensor}.

\subsection{Traditional Trend Mining}
We use SigniTrend \cite{schubert2014signitrend}. {\bf Micheal: Insert a third of
a column of description of signitrend here for self-containment. PLACEHOLDER
PLACEHOLDER PLACEHOLDER PLACEHOLDER PLACEHOLDER PLACEHOLDER PLACEHOLDER
PLACEHOLDER PLACEHOLDER PLACEHOLDER PLACEHOLDER PLACEHOLDER PLACEHOLDER
PLACEHOLDER PLACEHOLDER PLACEHOLDER PLACEHOLDER PLACEHOLDER PLACEHOLDER
PLACEHOLDER PLACEHOLDER PLACEHOLDER PLACEHOLDER PLACEHOLDER PLACEHOLDER
PLACEHOLDER PLACEHOLDER PLACEHOLDER PLACEHOLDER PLACEHOLDER PLACEHOLDER
PLACEHOLDER PLACEHOLDER PLACEHOLDER PLACEHOLDER PLACEHOLDER PLACEHOLDER
PLACEHOLDER PLACEHOLDER PLACEHOLDER PLACEHOLDER PLACEHOLDER PLACEHOLDER
PLACEHOLDER PLACEHOLDER PLACEHOLDER PLACEHOLDER PLACEHOLDER PLACEHOLDER
PLACEHOLDER PLACEHOLDER PLACEHOLDER PLACEHOLDER PLACEHOLDER PLACEHOLDER
PLACEHOLDER PLACEHOLDER PLACEHOLDER PLACEHOLDER PLACEHOLDER PLACEHOLDER
PLACEHOLDER PLACEHOLDER PLACEHOLDER PLACEHOLDER }

\subsection{Space Decomposition Scheme}
\begin{figure}[t] \centering \includegraphics[width =
0.44\columnwidth]{figures/spacedoof}
    \caption{k-t tree based space decomposition}
    \label{fig:spacedoof}
    \vspace{-0em}
\end{figure}
To fit a flow model between spatial regions, we need to minimize the bias that
results from having a non-uniform distribution of tweets on earth. We remedy
this problem by partitioning the geo-space in a way that minimized the
difference of tweets between spatial regions. For this purpose, we insert the
geo-locations of all tweets in our database into a k-d tree, having a maximum
node capacity of $1000$. Thus, every leaf node of this k-d tree is guaranteed to
have between $500$ and $1000$ two-dimensional points. Each of this leaf node is
then used as a spatial region in the remainder of the work. The decomposition
that we obtained this way is depicted in Figure \ref{fig:spacedoof}.


\subsection{Trend Flow Modeling}
\begin{figure}[t] \centering \includegraphics[width =
0.44\columnwidth]{figures/megadoof}
    \caption{Trend Flow Modelling}
    \label{fig:megadoof}
    \vspace{-0em}
\end{figure}
In this section we describe our approach of obtaining a trend flow from raw
trend counts. Thus, for a given trend, we consider all $N$ occurrences of this
trend at some time $t$ and all $M$ occurrences at the next time $t+1$. All the
regions having the trend at time $t$ can be considered as sources of the trend,
and all regions having the trend at time $t+1$ can be considered as targets of
the trend. Yet, we do not know any more specifically, which source region has
affected which target region and to what degree, since we do not know through
which channels and medias the trend was disseminated. Thus, due to lack of
better knowledge, we assume that all sources affect all target uniformly. This
flow model is formalized as follows
\begin{definition}[Spatio-Temporal Trend Flow Model]\label{def:tensor}
Let $\tau(K,T)$ be a trend having a set of keywords $K$ and having a time interval $T=\{T_1,...,T_{|T|}\}$ which covers $|T|$ epochs. Let $\mathcal{S}=\{S_1,...,S_{|\mathcal{S}|}\}$ be a space composition into $|\mathcal{S}|$ spatial regions. 
Furthermore, let $D(\tau_{K,T})_{i,j}=Occ(\tau_{K,T_i},S_j)$ be the trend count matrix of $\tau(K,T)$. We define the \emph{trend flow model} $F(\tau(K,T))$ of $\tau(K,T)$ as a $\mathcal{S}\times\mathcal{S}\times \{T_1,...,T_{|T-1|}\}$ tensor, such that
$$
F(\tau(K,T))_{i,j,k}=\frac{Occ(\tau_{K,T_k},S_i)\cdot Occ(\tau_{K,T_{k+1}},S_i)}{\sum{S_n\in\mathcal{S}}Occ(\tau_{K,T_{k+1}},S_i)}
$$
\end{definition}
Intuitively, an entry $F(\tau(K,T))_{i,j,k}$ of tensor $F(\tau(K,T))$ corresponds to the absolute flow of occurrences from region $S_i$ to region $S_j$ from time $T_k$ to time $T_{k+1}$. 
\begin{example}  
To illustrate the construction of tensor $F(\tau(K,T))$, consider an example
depicted in Figure \ref{fig:megadoof}. Here, the occurrences matrix of a tensor
of trend ??? is shown for four spatial regions. At the first point of time, the
trend appears twice in the first region and once in the fourth region, yielding the vector $(2,0,0,1)$. The the second and third point of time, this occurrence distribution is $(3,1,0,5)$ and $(2,1,3,4)$, respectively, yielding the trend count matrix shown in Figure \ref{fig:megadoof}. Transitioning from the first epoch $T_1$ to the second epoch $T_2$, the occurrences change from $(2,0,0,1)$ to $(3,1,0,5)$. The first spatial location $S_1$, having an initial of two tweets, is thus a source of the trend. Since we cannot observe the latent means of dissemination of a trend (through the internet, via TV, radio, word-of-mouth, etc.), we estimate that region $S_1$ disseminates its trend to all other regions having this trend. Since a fraction $\frac{3}{9}$ of all tweets at time $T_2$ are observed in region $S_1$, we estimate a trend-from of $\frac{2\cdot 3}{9}$ from region $S_1$ to itself. In contrast, only one trending tweet is observed at location $S_2$ at time $T_2$, of which we contribute a flow of $\frac{2\cdot 1}{9}$ to $S_1$. Similarly, a flow of $\frac{1\cdot 5}{9}$ is contributed from $S_4$ to $S_4$.
\end{example}
It is notable that each time-slice of tensor $F(\tau(K,T))$ is a rank-$1$
matrix, as all lines are multiples of each other. This redundancy is desirable, as it evenly distributes the flow from all source regions to all target regions, and this redundancy will be removed in a later tensor factorization step.
For each trend $\tau(K,T)$ we obtain a three-mode tensor as described in Definition \ref{def:tensor}. Concatenating these tensors for each trend $\tau\in\DB_\tau$ yields a four-mode tensor $\mathcal{F}(\DB)$ which is passed into the trend flow mining step described in the following.
\subsection{Trend Flow Mining}
\begin{figure}[t]
	\centering
\includegraphics[width = 0.44\columnwidth]{figures/tensordoof}
    \caption{Trend Flow Modelling}
    \label{fig:tensordoof}
    \vspace{-0em}
\end{figure}
We propose to decompose tensor $\mathcal{F}(\DB)$ using a Canonical / Parafac
tensor factorization \cite{?} using $k$ latent features, where $k$ is a parameter of our algorithm. As illustrated in Figure \ref{fig:tensordoof}, this factorization decomposes our four-mode $\mathcal{S}\times \mathcal{S} \times T \times \DB_\tau$ tensor into four two-mode matrices:
\begin{itemize}
\item a $\mathcal{S}\times k$ matrix containing a $k$-dimensional latent feature vector describing each source spatial region,
\item a $\mathcal{S}\times k$ matrix containing a $k$-dimensional latent feature vector describing each target spatial region,
\item a $T\times k$ matrix containing a $k$-dimensional latent feature vector describing each time epoch, and
\item a $\DB_\tau\times k$ matrix containing a $k$-dimensional latent feature vector describing each trend.
\end{itemize} 
These $k$-dimensional feature vectors can be used to identify mutually similar source spatial regions, mutually similar target spatial regions, mutually similar points in time, and mutually similar trends. 
\subsection{Trend Archetype Clustering}\label{subsec:trendclustering}
In our first mining step, we identify clusters of mutually similar trends, i.e., trends which have a similar feature vector after the factorization, and thus, since the tensor $\mathcal{F}(\DB)$ describes the flow of trends over time, exhibit a similar dissemination over space and time. Each of the resulting clusters is called a trend archetype. This approach allows to classify future trends among all archetypes, and allows to predict the future dissemination of a new trend by using the dissemination model of their archetype.
\begin{definition}
Let $\DB_\tau$ be a set of trends, and for each trend $\tau\in\DB_\tau$ let $\mbox{feat}(\tau)$ be a set of features describing $\tau$. Further, let $\mathcal{C}(\DB_\tau)=\{C_1,C_n\}$ be a clustering of all trends in $\DB_\tau$ into $n$ clusters. Then we denote each cluster $C\in\mathcal{C}$ as an archetype, and all trends $\tau\in C$ are said to belong to the same archetype.
\end{definition}

\subsection{Trend Archetype Flow Modelling}
After the trend clustering step of Section \ref{subsec:trendclustering}, we can identify sets of trends which belong to the same dissemination archetype. 
Therefore, we return to the full tensor $\mathcal{F}(\DB)$, and for each archetype $C\in\mathcal{C}$, we select only the trends $\tau\in C$, thus yielding a $\mathcal{S}\times\mathcal{S}\times T \times C$ tensor $\mathcal{F}(\DB,C)$ for each archetype $C$. Using $\mathcal{F}(\DB,C)$, we perform a projection on two modes $\mathcal{S}\times\mathcal{S}$ by averaging over all trends $\tau\in C$ and all epochs $T_i\in T$ to obtain the flow model of archetype $C$. 



\section{Experiments}
- Our tensor factorization exhibits a high quality even for low values of $k$. This inidicates large eigenvalues of the first $k$ latent features, thus indicating that these features are able to accurately describe the whole tensor with little loss of information. Having $k=4$ in our dataset, we observed a $49\%$ ??? measure compared to the undecomposed tensor. 
- We are able to obtain semantically meaningful clusters using our unsupervised clustering on the latent features of trends. For our Twitter dataset, our obtained clusters can be found in Table ?. The first cluster corresponds to ? trends, the second to ?, ..., ????. 
- We are able to predict the future dissemination of a trend, by classifiying its cluster type, and assuming a dissemination similar to the dissemination of its trend-archetype.

