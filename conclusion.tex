\vspace{-0.0cm}
\section{Conclusions}
\label{sec:conclusion}
Location-based social networks such as FourSquare, Yelp, TripAdvisor and OpenRice allow users to ``check-in'' and rate sites such as restaurants and hotels, to quantify their experience with that site. However, the vast majority of users does not use such online platforms to publish their opinion. To improve such user-site recommendation systems, we propose to exploit spatio-temporal data to estimate user-site ratings of any user. Therefore, we analysed trajectory data to find discriminative features indicating user preferences, such as their frequency of visit of a site, duration of visit, and distance from their home base. Using these features, we use existing (explicit) user-site recommendations to learn the relationship between our proposed features and the this ground-truth. Our experiments show, that our solution allows to drastically reduce the user-site-rating prediction error by exploiting spatio-temporal data. To leverage our solution to large spatio-temporal datasets, our next step is to automatically detect stay-points, rather than using pre-labelled check-in data. 